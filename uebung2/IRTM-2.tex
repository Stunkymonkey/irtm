\documentclass[12pt,pdftex,a4paper]{article}
\usepackage[ngerman]{babel}
\usepackage[utf8]{inputenc}
\usepackage{amsmath}
\usepackage{amssymb}
\usepackage{ulem}
%\usepackage{bbm}
\usepackage{array}
\usepackage{marvosym}
\usepackage{color}
\usepackage{hhline}
\usepackage[pdftex]{graphicx}
\usepackage{listings}
%\lstset{language=Python,basicstyle=\footnotesize}
\lstdefinestyle{base}{
	language=Python,
	emptylines=0,
	breaklines=true,
	basicstyle=\ttfamily\color{black},
	moredelim=**[is][\color{blue}]{@}{@},
}
\lstdefinestyle{output}{
	language=Python,
	emptylines=0,
	breaklines=true,
	basicstyle=\ttfamily\color{black},
	literate={ö}{{\"o}}1
			{ä}{{\"a}}1
			{ü}{{\"u}}1
}
\usepackage{pdfpages}
\usepackage{booktabs}
\PassOptionsToPackage{hyphens}{url}
\usepackage{hyperref}
\usepackage{xcolor}

\begin{document}
\title{ IRTM,\\ Homework 2}
\author{Dennis Tschechlov\\
		Felix Bühler\\
		Tobias Boceck\\
		Manuel Mergl}
\maketitle
\section*{Programming Task - SubTask 2}


New Lines are marked \textcolor{blue}{blue}!

\begin{lstlisting}[style=base]
#!/usr/bin/env python3

from collections import defaultdict
import numpy as np
@import sys
import re@

name = ""
inv_index = defaultdict(list)
correct_spelling = defaultdict()
postings = defaultdict(set)
suggestions = defaultdict(set)
stop_words = {'a', 'an', 'and', 'are', 'as', 'at', 'be', 'by', 'for', 'from',
              'has', 'he', 'in', 'is', 'it', 'its', 'of', 'on', 'that', 'the',
              'to', 'was', 'were', 'will', 'with'}
count = 0


@# if you look on the keyboard, these are the keys around every key (US-Layout)
key_q = {"w", "s", "a"}
key_w = {"q", "e", "a", "s", "d"}
key_e = {"w", "r", "s", "d", "f"}
key_r = {"e", "t", "d", "f", "g"}
key_t = {"r", "y", "f", "g", "h"}
key_y = {"t", "g", "h", "j", "u"}
key_u = {"y", "i", "h", "j", "k"}
key_i = {"u", "o", "j", "k", "l"}
key_o = {"i", "p", "k", "l"}
key_p = {"o", "l"}
key_a = {"q", "w", "s", "z", "x"}
key_s = {"q", "w", "e", "a", "d", "z", "x", "c"}
key_d = {"w", "e", "r", "s", "f", "x", "c", "v"}
key_f = {"e", "r", "t", "d", "g", "c", "v", "b"}
key_g = {"r", "t", "y", "f", "h", "v", "b", "n"}
key_h = {"t", "y", "u", "g", "j", "b", "n", "m"}
key_j = {"y", "u", "i", "h", "k", "n", "m"}
key_k = {"u", "i", "o", "j", "l", "m"}
key_l = {"i", "o", "p", "k"}
key_z = {"a", "s", "x"}
key_x = {"a", "s", "d", "z", "c"}
key_c = {"s", "d", "f", "x", "v"}
key_v = {"d", "f", "g", "c", "b"}
key_b = {"f", "g", "h", "v", "n"}
key_n = {"g", "h", "j", "b", "m"}
key_m = {"h", "j", "k", "n"}
@

def normalize(term):
    """
    normalize word and remove useless stuff
    """
    return term.lower().\
        replace(":", "").\
        replace(";", "").\
        replace(".", "").\
        replace(",", "").\
        replace("/", "").\
        replace("#", "").\
        replace("!", "").\
        replace("?", "").\
        replace("'", "").\
        replace("\"", "")


def index(filename):
    """
    indexes a given file and saves terms to a posting and
    non-positional inverted index
    """
    global name
    name = filename
    try:
        # open file
        with open(filename, "r") as file:
            docID = 0
            # iterate over each line in file
            for line in file:
                # split them to list of terms
                tweet = line.split()

                for term in tweet:
                    # remove clutter
                    term = normalize(term)
                    # check if term is in stop words to save some memory
                    if not str.isalpha(term) or (term in stop_words):
                        continue
                    # print(term)
                    @addSuggestions(term)@
                    # add docID
                    postings[term].add(docID)
                    # update inv_index
                    # we use the term as pointer, because python does not
                    # support pointers, and storing int by indexes will have an
                    # massive overhead
                    inv_index[term] = (len(postings[term]), term)
                # increase line number counter
                docID += 1
                # this is for displaying a progress while indexing
                if docID % 10000 == 0:
                    sys.stdout.write("\r{0} %".format(int(int(docID) / 10000)))
                    sys.stdout.flush()
                # if docID >= 100000:
                #     break
    except FileNotFoundError as e:
        raise SystemExit("Could not open file: " + str(e))
    return


def getLines(lines):
    """
    get lines of multiple lines
    """
    result = ""
    try:
        # open file
        with open(name, "r") as file:
            # iterate over the lines and print the lines, which match the terms
            for i, line in enumerate(file):
                if i in lines:
                    result += str(i) + "\t" + line
    except FileNotFoundError as e:
        raise SystemExit("Could not open file: " + str(e))
    return result


def query(term1, term2=""):
    """
    you can query your search terms. If only one term given it only searches
    for one, otherwise they both have to exist in the tweet
    """
    lines = []
    # remove clutter
    term1 = normalize(term1)
    term2 = normalize(term2)
    # if only one term given look for it
    if term1 in inv_index and not term2:
        (postings_len, postings_pointer) = inv_index[term1]
        # the sorted document_id list out of the postings_list
        lines = list(postings[postings_pointer])
    # if two terms are given look for both
    elif term1 in inv_index and term2 and term2 in inv_index:
        (postings_len, postings_pointer) = inv_index[term1]
        # the sorted document_id list out of the postings_list
        lines1 = sorted(list(postings[postings_pointer]))
        (postings_len, postings_pointer) = inv_index[term2]
        # the sorted document_id list out of the postings_list
        lines2 = sorted(list(postings[postings_pointer]))
        # init of iterators
        listiter1 = iter(lines1)
        listiter2 = iter(lines2)
        tmp1 = -1
        tmp2 = -1
        # and intersect the lists to see which lines match both terms
        # if the have the same lines, it will be added to 'lines'
        while True:
            try:
                # like discused in the lesson
                if tmp1 <= tmp2:
                    tmp1 = next(listiter1)
                else:
                    tmp2 = next(listiter2)
                if tmp1 == tmp2:
                    lines.append(tmp1)
            except StopIteration:
                break
    else:
        # if nothing is found it will look for alternative querys in the
        # suggestions dict
        print("nothing found")
        @# if simple query has nothing found
        if not term2:
            for new in suggestions[term1]:
                print("Possible search query: " + new)
                lines += query(new)
        # if complex query has nothing found for both querys
        elif term1 in suggestions and term2 in suggestions:
            for new1 in suggestions[term1]:
                for new2 in suggestions[term2]:
                    print("Possible search query: " + new1 + ", " + new2)
                    lines += query(new1, new2)
        # if complex query has nothing found for one query
        elif term1 in suggestions and term2 not in suggestions:
            print("Possible search query:")
            for new in suggestions[term1]:
                print("Possible search query: " + new + ", " + term2)
                lines += query(new, term2)
        # if complex query has nothing found for the other one query
        elif term1 not in suggestions and term2 in suggestions:
            print("Possible search query:")
            for new in suggestions[term2]:
                print("Possible search query: " + term1 + ", " + new)
                lines += query(term1, new)@
    # return lines
    return lines


@def read_correct(filename):
    """
    fill correct_spelling
    """
    try:
        # open file
        with open(filename, "r") as file:
            # iterate over the lines and add them to the correct_spelling
            for term in file:
                term = normalize(term)
                if term in stop_words:
                    continue
                if term not in correct_spelling:
                    correct_spelling[term] = 0
    except FileNotFoundError as e:
        raise SystemExit("Could not open file: " + str(e))@


@def addSuggestions(term):
    """
    fill suggestions but only generate worde with levinstein distance of one
    otherwise the programm will use to much ram
    with this configuration it will already use 33gb of ram
    """
    # iterate over the word by its lenght
    for i in range(0, len(term)):
        # get char at position i
        alpha = term[i]
        # if char ist not between a to z it skips to the next char
        if not re.match('[a-z]', alpha):
            continue
        # other wise it loads the keys around that char
        keys = eval('key_' + alpha)
        # it changes the char to every other value and stores it in suggestions
        for k in keys:
            new = list(term)
            new[i] = k
            suggestions[''.join(new)].add(term)
        # then it removes the one value and stores it
        new = list(term)
        del new[i]
        suggestions[''.join(new)].add(term)
        # adds key to the char (slipping of a key)
        new = list(term)
        for k in keys:
            new.insert(i, k)
            suggestions[''.join(new)].add(term)
        # adds keys to the position (dubble pressing)
        new = list(term)
        new.insert(i, alpha)
        suggestions[''.join(new)].add(term)@


@def levenshtein(A, B, thresh, insertion, deletion, substitution):
    """
    implementation of damerau-levenshtein from:
    https://gist.github.com/kylebgorman/1081951
    """
    D = np.zeros((len(A) + 1, len(B) + 1), dtype=np.int)
    for i in range(len(A)):
        D[i + 1][0] = D[i][0] + deletion
    for j in range(len(B)):
        D[0][j + 1] = D[0][j] + insertion
    for i in range(len(A)):  # fill out middle of matrix
        for j in range(len(B)):
            if A[i] == B[j]:
                D[i + 1][j + 1] = D[i][j]  # aka, it's free.
            else:
                D[i + 1][j + 1] = min(D[i + 1][j] + insertion,
                                      D[i][j + 1] + deletion,
                                      D[i][j] + substitution)
            if D[i + 1][j + 1] >= thresh:
                return None
    return D.item((-1, -1))@


if __name__ == '__main__':
    read_correct("english-words")
    # print("finished reading words")
    index("tweets")
    # print("\nfinished indexing")
    # print(len(inv_index))
    # for right exit code
    @try:
        # asking for more querys
        while True:
            # ask for input
            search = input('What are you looking for?: ')
            try:
                # if it is a simple query this is successfull
                term1, term2 = search.split(" ")
            except ValueError:
                # if the line before fails only this one will get executed
                # (simple query)
                print("simple search query")
                print(getLines(query(search)))
                continue
            # complex query
            print("tuple search query")
            print(getLines(query(term1, term2)))
    except KeyboardInterrupt:
        pass@

\end{lstlisting}

Sample Output:
\begin{lstlisting}[language=Python,style=output]

What are you looking for?: gelderm sejioren
tuple search query
nothing found
Possible search query: geldern, senioren
73940	2016-04-20 00:05:17 +0200	722546640238874625	@rpo_geldern	RP Online Geldern	Geldern - Senioren machen eine Reise zurück in die Kinderzeit https://t.co/57qoWYf6LO

What are you looking for?: gelix
simple search query
nothing found
Possible search query: helix
Possible search query: felix
Possible search query: relix
...

What are you looking for?: watterfall
simple search query
nothing found
Possible search query: waterfall
...

What are you looking for?: un stttgart
tuple search query
nothing found
Possible search query: uni, stuttgart
Possible search query: uno, stuttgart
Possible search query: uno, sttutgart
... and much more results


\end{lstlisting}


\end{document}
